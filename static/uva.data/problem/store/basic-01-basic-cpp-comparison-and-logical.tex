\subsubsection*{練習題}
\begin{itemize}[label={\Checkmark}]
\item \textbf{\textit{UVa \href{http://uva.onlinejudge.org/external/100/10055.html}{10055}: Hashmat the brave warrior}}\\
取絕對值有兩種做法,一種是用 \lstinline!if! 判斷;另一種是呼叫函數 \lstinline!abs()! 就好了。\lstinline!abs()! 函數被定義在 \lstinline!<cstdlib>! 中,雖然沒有 \lstinline!#include! 在 Visual C++ 依然能編譯過,但是上傳時因為編譯器的原因會導致\index{編譯錯誤}{\textbf{編譯錯誤}} (Compilation Error, CE)。\\另外要注意這一題的整數型態需用 \lstinline!long long!,用 \lstinline!int! 會造成「\index{溢位}{溢位現象}」,這個原因會在後面說明。
\item \textbf{\textit{UVa \href{http://uva.onlinejudge.org/external/111/11172.html}{11172}: Relational Operators}}\\
能夠理解題意就不難解決此道問題。
\item \textbf{\textit{UVa \href{http://uva.onlinejudge.org/external/119/11942.html}{11942}: Lumberjack Sequencing}}\\
依序給你一些鬍子的長度,問你這些鬍子是不是由長到短,或是由短到長排列。
\end{itemize}